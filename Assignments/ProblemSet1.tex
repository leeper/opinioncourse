\documentclass[a4paper]{exam}
\printanswers
\usepackage[margin=1in]{geometry}
\usepackage{hyperref}
\usepackage{booktabs}
\usepackage{alltt}
\usepackage{amsmath}

\title{Problem Set 1: Qualitative Interviewing}
\date{}

\begin{document}

\vspace{-4em}
\maketitle

\section{Purpose}\label{purpose}

The purpose of this problem set is to assess your understanding of one key method of qualitative public opinion research: small-group and individual interviewing.

\section{Overview}\label{overview}

You are asked to focus on one of the assigned readings for Week 1, examine the methods used in the text, and evaluate various aspects of those methods. 

\section{Your Task}\label{your-task}

\begin{enumerate}
\item Read the Conover, Searing, and Crewe text, ``The Elusive Ideal of Equal Citizenship'' \textit{Journal of Politics} 66(4): 1036--1068.

\item What is the research question in this article?

\begin{solution}

As we discussed in class, this article is concerned with: how do citizens in the US and UK define, understand, and make judgments about the definition of ``citizen'' or ``citizenship''? They ask a variety of secondary research questions such as, how do the US and UK differ in this respect? And how are definitions of citizenship shaped by urban/suburban/rural localities?

\end{solution}

\item What is the research design used in this study? What kinds of comparisons are intended by the authors and how do those comparisons help them to answer their research question?

\begin{solution}

The research design here has several components. First, there are two modes of data collection: qualitative focus groups emphasizing open-ended questions and large-$n$ surveys emphasizing quantitative analysis of closed-ended questions. Second, these methods of data collection are used to make several comparative analyses, with the most important being a US--UK cross-national comparison that aims to evaluate a political--cultural theory of how these two cases may differ. Third, several within-case and across-case comparisons are made between data generated from research participants in similar types of localities from the two cases.

While the data are certainly descriptively useful, the ability of those data (and corresponding analyses) to test the cultural theory posited by the authors is somewhat limited. While the US and UK are similar (and certainly, for example, urban areas in the two countries share additional commonalities), there are also numerous economic, institutional, and social differences between the two which cannot necessarily be addressed in a two-country comparison.

The choice of two methods is useful for the purposes of ``triangulation'' of results.

\end{solution}

\item How are focus groups used by the authors? How do they draw inferences relevant to their research question from the answers that participants supply during the focus groups? Is this evidence compelling, why or why not?

\begin{solution}

The focus groups are used primarily to generate descriptive claims about how citizens think about, judge, and articulate notions of citizenship. The paper is fairly vague about the actual procedures of selecting participants for focus groups and about how the focus groups themselves proceeded. Readers are not provided with full questionnaires nor full transcripts, so it is difficult to assess whether the researchers have ``cherry picked'' the quotes for the paper or whether these are generally characteristic of the statements made by participants.

The results might be more compelling if some of these details were provided. However, one must recall that the purpose of focus groups differs considerably from that of other methods (e.g., surveys) wherein the point is rarely to make generalizable claims about a phenomena (e.g., the prevalence of particular views or correlations between variables) but rather to identify and organize the language, frames, narratives, and discourses used by individuals. In this respect the article is somewhat successful.

\end{solution}

\item How are the quantitative survey data used by the authors? How do they draw inferences relevant to their research question from the answers that participants supply to closed-ended survey measures? Is this evidence compelling, why or why not?

\begin{solution}

The quantitative data are primarily used to make cross-national and between-locality comparisons. Specifically, the authors focus on generating statistics about the \textit{prevalence} of particular conceptualizations of citizenship. The statistical analysis rather simple --- closed-ended questions are asked that task respondents with categorizing their views among several available options and the proportions of respondents in each category are reported and compared. The analysis is compelling to the extent that it is fairly simple and transparent.

However, the collection of these data are not representative of the population of each country but rather of the communities in which the research was conducted. Thus this ``sample'' is what is called a ``cluster sample''. The authors do not report whether their analysis of these data accounts for the clustering of respondents within communities, so it would be reasonable to assume that it does not. This makes it somewhat problematic to make cross-national comparisons since the statistics labelled ``Britain'' and ``United States'' do not necessarily reflect those countries' populations as a whole.

\end{solution}

\item The text uses both ``open-ended'' questions (those allowing research participants to articulate answers in their own words) and ``closed-ended'' questions (those which require research participants to map their responses to questions a fixed set of predetermined response options). Discuss the relative merits of these questionnaire designs, with specific reference to ways that each were helpful or unhelpful for Conover et al.'s research.

\begin{solution}

There are numerous points that might be raised about open- and closed-ended questions. Generally, closed ended questions are seen as more useful for quantitative analysis because the data naturally fall into a set of categories that can be readily analyzed. To do the same with open-ended responses requires a process of ``coding'' the responses into categories that can then be analyzed. The primary strength of open-ended questions is that they allow respondents to articulate their ideas in their own words, which can be useful but also make it more difficult to analyze the results.

Open-ended questions are also subject to ``satisficing'' dynamics, wherein respondents may not be willing to put in the effort to provide a meaningful or extensive answer. This issue also applies to closed-ended questions, wherein respondents may provide ``don't know'' answers, leave questions blank, or consistently providing uninformative answers (always saying ``agree'', always ticking the first option, etc.).

In this particular paper, the two methods seem to be used quite well to ``triangulate'' on a set of inferences. The quantitative analysis of the closed-ended question data provide evidence of general patterns and the qualitative analysis of open-ended question data articulate those patterns in natural language.

The two methods appear to be used simultaneously. It may have been more useful to gather and analyze one form of data first before proceeding to the next, so as to inform the collection of the second data type with inferences drawn from the first. For example, conducting the focus groups first may have been useful for deciding precisely what closed ended questions to ask and with what response categories. Similarly, the closed ended questions on survey may have been useful to ask first before conducting focus groups so that the focus groups could be steered to clarifying the patterns (or lack thereof) seen in the quantitative analysis.

The authors could also have analyzed the open-ended responses using qualitative methods (e.g., calculating statistics like the percentage of participants that mention a particular qualification) but they chose not. It would have been helpful to know why they chose not to do this.

\end{solution}

\end{enumerate}

\section{Submission Instructions}\label{submission-instructions}

Please submit your answers as a PDF document via Moodle. It should be no more than 4 pages, single-spaced, in Times New Roman font size 12, on A4 paper with standard 2.54cm margins. This problem set is self-assessed. A solution set will be provided on the course website and the activity will be discussed in class.

\section{Feedback}\label{feedback}

Group feedback will be provided during class. If you would like more specific individual feedback on your work, please ask the instructor during office hours.

\end{document}
