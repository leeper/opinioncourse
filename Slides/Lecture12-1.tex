\documentclass[12pt]{beamer} %Makes presentation
\input{preamble}

\title{Public Opinion Matters?}

\date[]{November 20, 2013}

\begin{document}

\frame{\titlepage}
\frame{\tableofcontents}



\section{Gilens}

\frame{
\frametitle{Gilens}
\begin{itemize}\itemsep1em
\item Observed opinion versus ``informed'' opinion
\item Difficulty of changing information
\item Which should be represented?
\end{itemize}
}


\section{Representation}

\frame{
\frametitle{Representation}
\begin{itemize}\itemsep1em
\item Mansbridge
	\begin{itemize}
	\item Promissory
	\item Anticipatory
	\item Gyroscopic
	\item Surrogate
	\end{itemize}
\item Static opinion-policy congruence
\item Dynamic representation
\end{itemize}
}

\frame{
\frametitle{Mechanisms of influence}
\begin{itemize}\itemsep1em
\item Constituent interaction
\item Opinion polls
\item Interest groups
\end{itemize}
}

\frame<1>[label=groups]{
\frametitle{Group influence}
\begin{itemize}\itemsep1em
\item Do people form political groups?
	\begin{itemize}
	\item Truman
	\item Olson
	\end{itemize}
\item<2-> Do groups have influence?
	\begin{itemize}
	\item<2-> Jacobs and Page
	\item<2-> Schlozman
	\item<2-> Helboe Pedersen
	\end{itemize}
\end{itemize}
}


\frame{
\frametitle{Truman}
\begin{itemize}\itemsep1em
\item Group\only<2>{:\\ ``collection of individuals who have some characteristic in common'' (23)}
\item Interest group\only<4>{:\\ ``any group, that on the basis of one or more shared attitudes, makes certain claims upon other groups in the society for the establishment, maintenance, or enhancement of forms of behavior that are implied by the shared attitudes'' (33)}
\item Political interest group\only<6>{:\\ ``If and when it makes its claims through or upon any of the institutions of government, it becomes a political interest group'' (37)}
\item Latent groups
\end{itemize}
}

\againframe{groups}


\frame{
\frametitle{Two-way influence}
\begin{itemize}\itemsep1em
\item Disch's ``mobilization'' conception of representation
\item Schattschneider's ``scope of conflict'' principle
\vspace{1em}
\item<2-> Is there a tension between representation and self-interest?
\end{itemize}
}


\section{Riker}

\frame{
\frametitle{Representative versus Direct Democracy}
\begin{itemize}\itemsep2em
\item James Madison on size of republics
\item If people voted directly on policy questions, what would that look like
\item Riker Ch. 1
\end{itemize}
}

\frame{
\frametitle{Attitudes versus Preferences}
\begin{itemize}\itemsep1em
\item Attitudes are not comparative
\item Can people form (transitive) preference rankings?
\end{itemize}
}


% Talk about Condorcet?

\frame{
\frametitle{Median voter theorem}
\begin{itemize}\itemsep1em
\item Majority rule in one dimension with two alternatives
\item Majority rule in two dimensions
	\begin{itemize}
	\item No median
	\item Cycling
	\item Institutions matter (esp. agenda control)
	\end{itemize}
\item Majority rule with more than two alternatives
\end{itemize}
}


\frame{
\frametitle{Riker's (Arrow's) ``fairness'' principles (Ch. 5)}
\begin{itemize}\itemsep1em
\item Universal admissibility of preferences
\item Non-dictatorship
\item Independence of irrelevant alternatives (IIA)
\item Pareto Optimality
	\begin{itemize}
	\item Monotonicity
	\item Non-imposition
	\item Instead: Preferred alternatives should be ranked higher in aggregate
	\item Similar to a Condorcet criterion
	\end{itemize}
\end{itemize}
}


\frame{
\frametitle{Cycling and Paradoxes}
\Large
\only<2>{A vs B}
\only<3>{B vs C}
\only<4>{A vs C}
\only<6>{Plurality}
\only<7-8>{Majority/Run-off}
\begin{center}
\textbf<2,4,6,7-8>{A} > \textbf<3>{\emph<8>{B}} > C \visible<5->{\hspace{3em}(9)}\\
\vspace{2em}
\textbf<2,3>{\emph<8>{B}} > \textbf<4>{C} > A \visible<5->{\hspace{3em}(8)}\\
\vspace{2em}
\textbf<3,4>{C} > \textbf<2,7-8>{A} > B \visible<5->{\hspace{3em}(7)}
\end{center}
}

\frame{
\frametitle{Implications}
\begin{itemize}\itemsep2em
\item Even if individual preferences are well-defined, no guarantee of a coherent, fair social choice
\item Testing for opinion-policy congruence seems easy, but not if we actually use a populist institution as baseline
\item What if we can't measure preferences? 
\item And what if people don't have preferences at all? 
\end{itemize}
}




% REWRITE PREVIEW SLIDES


\frame{}

\end{document}
