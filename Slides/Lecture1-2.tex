
<!-- saved from url=(0072)https://raw.github.com/leeper/opinioncourse/master/Slides/Lecture1-2.tex -->
<html><head><meta http-equiv="Content-Type" content="text/html; charset=UTF-8"><style type="text/css"></style></head><body><pre style="word-wrap: break-word; white-space: pre-wrap;">\documentclass[12pt]{beamer} %Makes presentation
%\documentclass[handout]{beamer} %Makes Handouts
\usetheme{Singapore} %Gray with fade at top
\useoutertheme[subsection=false]{miniframes} %Supppress subsection in header
\useinnertheme{rectangles} %Itemize/Enumerate boxes
\usecolortheme{seagull} %Color theme
\usecolortheme{rose} %Inner color theme

\definecolor{light-gray}{gray}{0.75}
\definecolor{dark-gray}{gray}{0.55}
\setbeamercolor{item}{fg=light-gray}
\setbeamercolor{enumerate item}{fg=dark-gray}

\setbeamertemplate{navigation symbols}{}
\setbeamertemplate{mini frames}[default]
\setbeamercovered{dynamics}
\setbeamerfont*{title}{size=\Large,series=\bfseries}

%\setbeameroption{notes on second screen} %Dual-Screen Notes
%\setbeameroption{show only notes} %Notes Output

\newcommand{\heading}[1]{\noindent \textbf{#1}\\ \vspace{1em}}

\usepackage{bbding,color,multirow,times,ccaption,tabularx,graphicx,verbatim,booktabs,fixltx2e}
\usepackage{colortbl} %Table overlays
\usepackage[english]{babel}
\usepackage[latin1]{inputenc}
\usepackage[T1]{fontenc}

\title{Preview of\\``What is Public Opinion?''}

%\author[]{Thomas J. Leeper}
\institute[]{
  \inst{}%
  Department of Political Science and Government\\Aarhus University
}
\date[]{September 3, 2013}

\begin{document}

\frame{\titlepage}

\frame{
\heading{Overview for next week}
}


\frame&lt;4&gt;[label=readings]{
\heading{Readings for Next Week}
\begin{itemize}\itemsep1em
\item Dahl -- {\em Democracy and Its Critics}
\item&lt;2-&gt; Herbst -- {\em Numbered Voices}
\item&lt;3-&gt; Downs -- {\em An Economic Theory of Democracy}
\item&lt;4-&gt; Mansbridge -- ``Rethinking Representation''
\end{itemize}
}

\againframe&lt;1&gt;{readings}

\frame{
\heading{Dahl}
\begin{itemize}\itemsep1em
\item Under what conditions are citizens fit to self-govern?
\item How does/should democracy work?
\item How must citizens act in democracy?
\item Whose voice should be heard in democracy?
\end{itemize}
}

\againframe&lt;2&gt;{readings}

\frame{
\heading{Herbst}
\begin{itemize}\itemsep1em
\item What is ``public opinion''?
\item How do we know public opinion?
\item History of public opinion research
\end{itemize}
}

\againframe&lt;3&gt;{readings}

\frame{
\heading{Downs}
\begin{itemize}\itemsep1em
\item An economic model of how democracy works
\item Assumptions about citizens that underly that model
\item What are citizens supposed to do in democracy?
\item How do citizens vote rationally?
\end{itemize}
}

\againframe&lt;4&gt;{readings}

\frame{
\heading{Mansbridge}
\begin{itemize}\itemsep1em
\item Four models of representation
	\begin{itemize}
	\item Promissory
	\item Anticipatory
	\item Gyroscopic
	\item Surrogate
	\end{itemize}
\item How does the public influence their representatives in each model?
\item What does each model imply about democracy?
\end{itemize}
}


\frame{
\heading{Who will lead discussion next week?}
}

\appendix
\frame{}

\end{document}
</pre></body></html>