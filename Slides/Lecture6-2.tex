\documentclass[12pt]{beamer} %Makes presentation
\input{preamble}

\title{Preview of\\``Are opinions responsive?''}

\date[]{October 9, 2013}

\begin{document}

\frame{\titlepage}

\frame{
\heading{Overview for two weeks from today}

No class on October 16
}


\frame{
\heading{Syllabus change}
\begin{itemize}
\item Optional: Lodge et al. -- ``The Responsive Voter''
\item Refresher on online (OL) models
\end{itemize}
}

\frame<3>[label=readings]{
\heading{Readings for Two Weeks from Today}
\begin{itemize}\itemsep1em
\item<1-> Taber and Lodge -- ``Motivated Skepticism''
\item<2-> Page et al. ``What Moves Public Opinion?''
\item<3-> Healy -- ``Random Events, Economic Losses, and Retrospective Voting''
\end{itemize}
}

\againframe<1>{readings}

\frame{
\heading{Taber and Lodge}
\begin{itemize}\itemsep1em
\item What is motivated reasoning? (What is reasoning? What is motivation?)
\item What kinds of processes are associated with motivated reasoning?
\item What are the implications of citizens as motivated reasoners? Is this good or bad for democracy?
\item Are opinions formed through motivated reasoning ``quality'' opinions? Are they rational?
\end{itemize}
}

\againframe<2>{readings}

\frame{
\heading{Page et al.}
\begin{itemize}\itemsep1em
\item How do Page et al. think of ``public opinion''?
\item What is their analytic strategy? How does it differ from what we've seen before?
\item How well do their results generalize beyond time, context, and geography?
\end{itemize}
}

\againframe<3>{readings}

\frame{
\heading{Healy and Malhotra}
\begin{itemize}\itemsep1em
\item They look at the effect of tornadoes on voting behavior
\item What is their research design and what do they find?
\item What do their results say about voter rationality?
\end{itemize}
}


\frame{
\heading{Who will write papers for next time (in two weeks)?}
}

\appendix
\frame{}

\end{document}
