\documentclass[12pt]{beamer} %Makes presentation
%\documentclass[handout]{beamer} %Makes Handouts
\usetheme{Singapore} %Gray with fade at top
\useoutertheme[subsection=false]{miniframes} %Supppress subsection in header
\useinnertheme{rectangles} %Itemize/Enumerate boxes
\usecolortheme{seagull} %Color theme
\usecolortheme{rose} %Inner color theme

\definecolor{light-gray}{gray}{0.75}
\definecolor{dark-gray}{gray}{0.55}
\setbeamercolor{item}{fg=light-gray}
\setbeamercolor{enumerate item}{fg=dark-gray}

\setbeamertemplate{navigation symbols}{}
\setbeamertemplate{mini frames}[default]
\setbeamercovered{dynamics}
\setbeamerfont*{title}{size=\Large,series=\bfseries}

%\setbeameroption{notes on second screen} %Dual-Screen Notes
%\setbeameroption{show only notes} %Notes Output

\newcommand{\heading}[1]{\noindent \textbf{#1}\\ \vspace{1em}}

\usepackage{bbding,color,multirow,times,ccaption,tabularx,graphicx,verbatim,booktabs,fixltx2e}
\usepackage{colortbl} %Table overlays
\usepackage[english]{babel}
\usepackage[latin1]{inputenc}
\usepackage[T1]{fontenc}

\title{Preview of\\``Are opinions constrained?''}

%\author[]{Thomas J. Leeper}
\institute[]{
  \inst{}%
  Department of Political Science and Government\\Aarhus University
}
\date[]{September 11, 2013}

\begin{document}

\frame{\titlepage}

\frame{
\heading{Overview for next week}
}

\frame{
\heading{Some Background}
\begin{itemize}\itemsep1em
\item Eagly and Chaiken talked about inter-attitudinal structure as an empirical question
\item Political scientists often see inter-attitudinal structure as normatively important
\item Is structure important? If so, why?
\item What are the democratic implications of structured or unstructured attitudes?
\end{itemize}
}


\frame<4>[label=readings]{
\heading{Readings for Next Week}
\begin{itemize}\itemsep1em
\item Feldman -- ``Structure and Consistency in Public Opinion''
\item<2-> Jost et al. -- ``Political Ideology''
\item<3-> Smith et al. -- ``Biology, Ideology, and Epistemology''
\end{itemize}
}

\againframe<1>{readings}

\frame{
\heading{Feldman}
\begin{itemize}\itemsep1em
\item What are political values? Are they universal?
\item Are values something different from attitudes or ideology?
\item Feldman uses values to explain opinions and retrospective evaluations. If stable values explain retrospective evaluations of government, what implications are there for representation?
\end{itemize}
}

\againframe<2>{readings}

\frame{
\heading{Jost et al.}
\begin{itemize}\itemsep1em
\item What is ideology? Is it important for citizens to be ideological?
\item Where does ideology come from? From politics? From biology, socialization, or something else?
\item How does ideology compare to opinions? Are they the same thing?
\end{itemize}
}

\againframe<3>{readings}

\frame{
\heading{Smith et al.}
\begin{itemize}\itemsep1em
\item There are two common approaches to studying the genetic basis of opinions. This article addresses one of them (twin studies) and we'll talk about the other technique (candidate genes studies) in class next week.
\item How does a twin study work and how does it test whether opinions are genetic? Is this approach credible, why or why not?
\item What implications for democracy (e.g., representation) are there if opinions are genetically transmitted? Are these implications different from or the same as those if opinions are socialized from parent to child?
\end{itemize}
}


\frame{
\heading{Who will lead discussion next week?}
}

\frame{}

\end{document}
