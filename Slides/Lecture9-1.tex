\documentclass[12pt]{beamer} %Makes presentation
\input{preamble}

\title{Do people act on their opinions?}

\date[]{October 30, 2013}

\begin{document}

\frame{\titlepage}
\frame{\tableofcontents}

\section{Assessment}
\frame{\tableofcontents[\thesection]}

\frame{
\frametitle{Survey from last week}
\begin{itemize}\itemsep1em
\item You seem to enjoy the course
\item A bit too much (difficult) reading
\item Desire for more lecture and structure
\item More activities to prep for the exams
\end{itemize}
}

\frame{
\frametitle{Writing activity from last week}
\begin{itemize}\itemsep1em
\item Was this helpful to you?
\item Would you like to do it again?
\end{itemize}
}

\section{Economic Models}

\frame{
\frametitle{Economic models}
\begin{itemize}\itemsep1em
\item Downsian tradition
\item Riker and Ordeshook
\item Olson
\end{itemize}
}

\frame{
\frametitle{Economic models: Riker and Ordeshook}
\begin{itemize}\itemsep1em
\item Why/when do people vote?
\item $R = PB - C + D$
\item How valid is their model?
\end{itemize}
}

\frame{
\frametitle{Economic models: Olson}
\begin{itemize}\itemsep1em
\item Why do people participate (in groups)?
\item Group interests
\item What is he arguing against?
\item Collective action problem
\end{itemize}
}


\section{Psychological Models}

\frame{
\frametitle{Psychological Models}
\begin{itemize}
\item Eagly and Chaiken
\item Fazio and Towles-Schwen
\end{itemize}
}

\frame{
\frametitle{Psychological Models}
\begin{itemize}
\item ``Expectancy value'' models
	\begin{itemize}
	\item Theory of reasoned action
	\item Theory of planned behavior
	\item Eagly and Chaiken's composite model
	\end{itemize}
\item Automatic activation models
	\begin{itemize}
	\item MODE model
	\end{itemize}
\end{itemize}
}



\section{Exercise}

\frame{
\frametitle{Short Writing Exercise}
\begin{itemize}\itemsep1em
\item Write for 5 minutes
\item \emph{Discuss the following:}\\
One tradition views opinions and behaviors as simple, rational cost-benefits calculations (i.e., people prefer and do what maximizes their utility). Another view says that we need to understand the psychology of opinion formation and opinion-behavior relations in order to make sense of public opinion. Which view is more correct? Why?
\item Some ideas:
	\begin{itemize}
	\item Rationality and irrationality
	\item Constraint
	\item Information processing (OL/MB)
	\item Nonattitudes
	\item Attitudes vs. Preferences
	\item Etc.
	\end{itemize}
\end{itemize}
}

\frame{
\frametitle{Short Writing Exercise}
\begin{itemize}\itemsep1em
\item Form groups of three
\item Each person share their ideas
\item Attempt to reach a consensus
\end{itemize}
}

\frame{}

\end{document}
