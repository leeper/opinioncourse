\documentclass[12pt]{beamer} %Makes presentation
\input{preamble}

\title{Preview of\\``What shapes public opinion?''}

\date[]{October 2, 2013}

\begin{document}

\frame{\titlepage}

\frame{
\heading{Overview for next week}

Moving into the second part of the course\dots
}


\frame<4>[label=readings]{
\heading{Readings for Next Week}
\begin{itemize}\itemsep1em
\item Petersen et al. -- ``Deservingness versus Values in Public Opinion on Welfare''
\item<2-> Gerber et al. ``How Large and Long-Lasting Are the Persuasive Effects of Televised Campaign Ads?''
\item<3-> Chong and Druckman -- ``Framing Public Opinion in Competitive Democracies''
\end{itemize}
}

\againframe<1>{readings}

\frame{
\heading{Petersen et al.}
\begin{itemize}\itemsep1em
\item What is the deservingness heuristic? Where does it come from?
\item What does this study say about opinion constraint? (Especially about values)
\item If opinions about welfare are about deservingness, what role is their for political debate/deliberation in changing opinions?
\end{itemize}
}

\againframe<2>{readings}

\frame{
\heading{Gerber et al.}
\begin{itemize}\itemsep1em
\item What is their research design? What is the treatment?
\item Why did this study get published in the top political science journal?
\item What does it tell us about how people form opinions, if anything?
\item What does the study say about elite influence on opinion?
\item What does it say about the stability of opinions over time?
\end{itemize}
}

\againframe<3>{readings}

\frame{
\heading{Chong and Druckman}
\begin{itemize}\itemsep1em
\item What does political debate look like?
\item What is ``framing''?
\item How do opinions respond to different kinds of political debate?
\item How realistic is the research design? How credible are their results?
\end{itemize}
}


\frame{
\heading{Who will lead discussion next week?}
}

\appendix
\frame{}

\end{document}
