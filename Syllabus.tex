\documentclass[12pt,a4paper]{article}
\usepackage[top=1in, bottom=1in, left=1in, right=1in]{geometry}
\usepackage{graphicx,setspace,hyperref,amsmath,amsfonts,multirow,ccaption,mdwlist,comment}
% mini table of contents
\usepackage{minitoc}
\dosecttoc % make section toc
\setcounter{secttocdepth}{2} % subsection depth
\renewcommand{\stctitle}{} % no title
\nostcpagenumbers

% optionally include commented environments
\excludecomment{lessonplan}
\excludecomment{finalexam}

\setlength{\marginparwidth}{.5in}
\usepackage{natbib}
% Two lines to create in-text full citations for a syllabus
% And comment out my other standard bibtext commands
\usepackage{bibentry}
\newcommand{\reading}[2][]{\noindent -- {#1 }\bibentry{#2}.\vspace{.25em}\\}
\newcommand{\seealso}{\noindent \emph{See Also:}\\}
\newcommand{\topic}[1]{\noindent \textbf{#1}\\}
\usepackage[T1]{fontenc}
\usepackage{lmodern}
\hypersetup{
    bookmarks=true,         % show bookmarks bar?
    unicode=false,          % non-Latin characters in Acrobat’s bookmarks
    pdftoolbar=true,        % show Acrobat’s toolbar?
    pdfmenubar=true,        % show Acrobat’s menu?
    pdffitwindow=false,     % window fit to page when opened
    pdfstartview={FitH},    % fits the width of the page to the window
    pdftitle={Syllabus: Does Public Opinion Matter?},    % title
    pdfauthor={Thomas J. Leeper},     % author
    pdfsubject={Political Science},   % subject of the document
    pdfkeywords={politics} {public opinion} {political psychology}, % list of keywords
    pdfnewwindow=true,      % links in new window
    pdfborder={0 0 0}
}

\title{Does Public Opinion Matter?}
\author{Thomas J. Leeper\\
Department of Political Science and Government\\
Aarhus University}

\begin{document}
\nobibliography*

\maketitle

\faketableofcontents

\section{Overview}

Does public opinion matter? At its core, democracy is often defined as government by the people. Through centuries of philosophical writing, the democratic idea has been defended as the best form of government relative to all alternatives due, in part, to its equal incorporation of individuals' views. The aggregation of the public's preferences through polling, elections, and other forms of political participation is seen as normatively superior to oligarchy, anarchy, or absolutist forms of government. Yet, most citizens are not directly part of the public policy process and considerable skepticism exists regarding the capacity of citizens to form and act on their political opinions. How, then, do citizens' views get represented? Should they be represented? Are democratic institutions able to reflect public views effectively and translate those views into policy? And are the public's preferences --- individually or in aggregate --- sufficiently informed, meaningful, coherent, stable, and/or responsive to external events to merit being the basis for the democratic form of government? The course is structured as a dialog between normative perspectives and empirical research on public opinion formation, representation, and political influence, as well as a debate between views about the importance, value, and relevance of public opinion and skeptical positions about the incoherence or irrelevance of public opinion in contemporary democratic politics.

Public opinions determine which parties, candidates, and referenda win, but government activity itself is rarely shaped by the public's views. Public opinions are deeply constrained by stable individual-differences, such as biology, ideology, and values, but at the same time show instability over-time and reflect often incoherent belief systems. The majority public's views on different issues are often responsive to changing political events, but this direction of causality runs opposite of the opinion-policy linkage demanded by most theories of representation. While opinions motivate citizens' engagement with politics, they also bias the form of that engagement and many people remain disinterested in fulfilling the demands of citizenship. Government policies that run contrary to public views violate democratic ideals of government by the people, but citizens often hold views that are ignorant, racist, xenophobic, or otherwise harmful. If public opinion is the centerpiece of democratic government, why is it such a problematic concept in contemporary politics?

The purpose of this course is to explore issues related to public opinion --- what opinions are and how they are formed, how opinions shape citizens' political behavior, and how legislatures and other governmental institutions respond (or do not respond) to citizens' preferences. Students will leave the course with a thorough theoretical understanding of political opinions, their origins, and their possible effects through exposure to philosophical perspectives, contemporary case studies, and a broad set of empirical research. The course will challenge assumptions about what democracy is and how it works, but will also provide students with insight into how government --- in legislative, judicial, and bureaucratic capacities --- should work and what role public servants have in influencing and responding to the public's views.

\section{Practical Matters}

The expectations for this course are that students (1) participate actively, regularly, and positively in classroom discussions (which will constitute the bulk of the course's content), (2) lead discussion on at least one day of class, and (3) complete a written exam answering questions raised by the course using relevant theoretical and empirical literature. Toward the first expectation, students should read the assigned reading ahead of the day on which they are assigned and have at least two questions in mind that were provoked by those readings that might be answered in class or serve as a topic for discussion. Toward the second end, students will sign up for one or more weeks to lead discussion (on the first day of class), which will also involve writing a short (one-page) response essay to structure that discussion. On their assigned week, students can structure class discussion however they so choose, but should use submitted discussion questions where useful.

\section{Objectives and Evaluation}
After this course, students should be able to:
\begin{enumerate*}
\item Explain what opinions are, how they are formed, and how they behave.
\item Apply knowledge of opinions and opinion measurement to the evaluation of survey public opinion research.
\item Explain different conceptualizations of political representation and their empirical implications.
\item Apply theories of representation to the evaluation of public processes and institutions.
\item Evaluate arguments about the proper role of public opinion in democracy and government.
\end{enumerate*}

\clearpage
\section{Course Outline}
The general schedule for the course is as follows. Details on the readings for each week are provided on the following pages.
\secttoc
\clearpage

\subsection{Introduction and Course Overview (4 Sep)}
No assigned reading

\begin{lessonplan}
\subsubsection{Lesson Plan}
	\begin{itemize*}
	\item Go around; get to know everyone
	\item Discuss the following:
		\begin{itemize*}
		\item We will wrestle with a series of questions related to our big question: ``Does Public Opinion Matter?''
		\item Start off with some normative foundations and psychological principles
		\item Explore individual-level questions about whether public opinion can matter
		\item Next we'll do a little bit of bridge work connecting these micro foundations to macro dynamics of public opinion
		\item Then we'll look at institutional-level questions about how public opinion matters
		\item We will end by trying to integrate everything we've looked at to make an overall assessment of the role of the public in contemporary politics
		\end{itemize*}
	\end{itemize*}
\end{lessonplan}

\subsection{Theory -- What is public opinion? (11 Sep)}
%\reading[Selections from]{Dahl1971}
\reading[Chapters 7--9 from]{Dahl1989}
\reading[Chapters 2--3 from ]{Herbst1995} % Numbered Voices
\reading[Chapters 1--3 from ]{Downs1957}
%\reading[Number X from ]{FederalistPapers}
\reading{Mansbridge2003}
%\reading{Rousseau}
%\reading[Chapter 1 from]{Riker1988}
%\reading[Selections from ]{Pitkin1967}

\begin{lessonplan}
	\begin{itemize*}
	\item talk about core ideas of power and choice:
	\item everyone wants power
	\item democracy is the choice of who gets power
	\item public opinion is supposed to guide choices, which in turn determine distribution of power and resources
	\end{itemize*}
\end{lessonplan}

\subsection{Psychology I -- How do people form opinions? (18 Sep)}
\reading{DruckmanLupia2000}
\reading{EaglyChaiken1998}
\reading[Chapter 3 from ]{Zaller1992}
%\reading{ZallerFeldman1992}
%\reading{LodgeMcGraw1995}
%\reading{JenningsNiemi1975}
% Jennings, M. Kent, Laura Stoker, and Jake Bowers. 2009. “Politics Across Generations: Family Transmission Reexamined.” Journal of Politics 71(3): 782-799.
\reading{LauRedlawsk2001}

\begin{lessonplan}
Talk about misinformation and \reading{Kuklinskietal2000}

\end{lessonplan}


\subsection{Psychology II -- Are opinions constrained? (25 Sep)}
%\reading[Selections about ideology and constraint from ]{Converse1964}
\reading{JostFedericoNapier2009}
\reading{Feldman1988} % or maybe chapter from PolPsy HB
%\reading{Lacy2001a}
%\reading{Druckman2001a}
\reading{Smithetal2012}


\begin{lessonplan}
Talk about \reading{CharneyEnglish2012} % candidate genes article

\end{lessonplan}



%\subsection{Psychology III -- Does information matter? (25 Sep)}
%\reading[Chapters 6-7 from ]{DelliCarpiniKeeter1997}
%\reading{GronlundMilner2006}
%\reading{TilleyWlezien2008}
\begin{lessonplan}
Tables on comparative political knowledge from \citet[90-92]{DelliCarpiniKeeter1997}

Talk about \reading{GronlundMilner2006}

\end{lessonplan}


\subsection{Methods -- How to measure opinion? (2 Oct)}
\reading{TourangeauRasinski1988}
\reading{BishopTuchfarberOldendick1986}
\reading{SchuldtKonrathSchwarz2011}
%\reading{Davis1997}
% More about measurement problems, social desirability

\noindent {\em Optional reading:}\\
\reading{YeagerLarsonKrosnickTompson2010}
\reading{Krosnicketal2002}


\begin{lessonplan}
Have people conduct a cognitive survey interview over the break
Talk about interviewer effects
\end{lessonplan}


\subsection{Influences I -- What shapes public opinion(s)? (9 Oct)}

%\topic{Media}
%\reading{Iyengar1990}
\reading{Petersenetal2011}
\reading{Gerberetal2011}
%\topic{Interpersonal influence}
%\reading{MutzImpersonalInfluence}
%\reading{HuckfeldtSprague}
%\reading{Nickerson2008}
\reading{ChongDruckman2007a}


\subsection{{\em No class meeting} (16 Oct)}
\vspace{2em}


\subsection{Influences II -- Are opinions responsive? (23 Oct)}
%\reading{LodgeSteenbergenBrau1995}
\reading{TaberLodge2006}
\reading{PageShapiroDempsey1987}
%\reading{Holbrooketal2005}
\reading{Healy2010}
%\reading{BennettIyengar2008}

\begin{lessonplan}
History of media effects: Propaganda versus public service messages
Talk about selective exposure
\end{lessonplan}


\subsection{Participation I -- Do people act on their opinions? (30 Oct)}
%\reading{CitrinGreen1990}
\reading[Selections from]{EaglyChaiken1998}
\reading{FazioTowlesSchwen1999}
\reading{RikerOrdeshook1968}
\reading[Chapters 1--2 from]{Olson1965}
%\reading[Selection on issue publics from ]{Converse1964}
%\reading{Krosnick1990} % issue publics
%\reading{EatonVisser2008}
%\reading{MillerKrosnick2004}
%\reading[Ch 14 from]{Downs1957}
%\reading[Chapters 2,4,6 from ]{Campbelletal1960}

\begin{lessonplan}
Talk about issue publics
Talk about Miller and Krosnick study
\end{lessonplan}


\subsection{Participation II -- Do campaigns help citizens? (6 Nov)}
% Pol comm; campaigns (American Voter; Lazarsfeld)
%\reading{PettyCacioppo1986}
%\reading{Kinder2003}
%\reading{RabinowitzMacDonald1989}
%\reading{Fiorina1980}
%\reading{Lewis-Beck1986}
\reading{LauRedlawsk1997}
\reading{Sniderman2000} % from Elements of Reason
\reading{Hobolt2007}
%\reading{SidesCitrin2007}
%\reading{LodgeMcGrawStroh1989}
%\reading{SchuckdeVreese2008}
%\reading{IyengarSimon2000}
%\reading{Selbetal2009}

\noindent {\em Optional reading:}\\
\reading{Disch2011}



\subsection{Aggregation -- From Micro to Macro? (13 Nov)}
%\reading[Selection about nonattitudes from]{Converse1964}
\reading{DruckmanLeeper2012b} % macro-micro
%\reading{LechelerdeVreese2011}
\reading[Chapter 10 from]{PageShapiro1992}
\reading{MacKuenEriksonStimson1989} % Macropartisanship
\reading{Gilens2001}

\begin{lessonplan}
Review survey interviews from over the break
\end{lessonplan}





\subsection{Representation I -- Public opinion matters? (20 Nov)}
%\reading{Erikson1978}
\reading{JacobsPage2005}
%\reading{PageShapiro1983}
\reading{StimsonMacKuenErikson1995}
\reading[Chapters 1 and 5 from]{Riker1988}
%\reading{McGuireStimson2004}
%\reading{McCloskyHoffmannOhara1960}



\subsection{Representation II -- Groups matter? (27 Nov)}
% Dahl Who Governs?
%\reading[Chapters 1--2 from]{Schattschneider1975}
%\reading{DenzauMunger1986}
\reading[Chapter 2 from]{Truman1971}
\reading[Revisit Chapters 1--2 from]{Olson1965}
\reading{Schlozman1984}
% ONE OF THE FOLLOWING:
%\reading{Smith1995}
\reading{Pedersen2013} % Helene's article on interest group influence

\begin{lessonplan}
Talk about Schattschneider
Use Denzau and Munger as case study
\end{lessonplan}

\subsection{Opinion and Representation -- Theory and practice? (4 Dec)}
%\reading{Lippman1922}
%\reading{Lippman1928}
%\reading{Dewey}
%\reading{Dahl2006}
\reading[Chapter 10 from]{Riker1988}
\reading[Chapters 7-8 from]{HibbingTheiss-Morse2002}
%\reading{Barabas2004}
\reading[Chapter 2 (provided digitally) from]{Snidermanetal2014}

\noindent {\em Optional additional reading:}\\
\reading{BachrachBaratz1962}
\reading{SullivanPiersonMarcus1979}
%\reading{GibsonGouws2001}
%\reading{SullivanTransue1999}
%\reading{HurwitzPeffley2008}
\reading{Huddyetal2005}
\reading{Fishkin2006}
\reading[Chapter 8 from]{Schattschneider1975}
\reading{UrbinatiWarren2008}


% load bibtext, but don't generate a bibliography
\bibliographystyle{plain}
\nobibliography{Syllabi}

\end{document}