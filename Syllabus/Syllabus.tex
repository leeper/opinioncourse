\documentclass[12pt,a4paper]{article}
\usepackage[margin=1in]{geometry}
\usepackage{graphicx,setspace,hyperref,amsmath,amsfonts,multirow,ccaption,mdwlist,comment}

\usepackage{minitoc}
\dosecttoc
\setcounter{secttocdepth}{2}
\renewcommand{\stctitle}{}
\nostcpagenumbers

\setlength{\marginparwidth}{.5in}
\usepackage{natbib}
\usepackage{bibentry}
\newcommand{\reading}[2][]{\noindent -- {#1 }\bibentry{#2}.\vspace{.25em}\\}
\newcommand{\seealso}{\noindent \emph{See Also:}\\}
\newcommand{\topic}[1]{\noindent \textbf{#1}\\}
\usepackage[T1]{fontenc}
\usepackage{lmodern}
\hypersetup{
    bookmarks=true,         % show bookmarks bar?
    unicode=false,          % non-Latin characters in Acrobat’s bookmarks
    pdftoolbar=true,        % show Acrobat’s toolbar?
    pdfmenubar=true,        % show Acrobat’s menu?
    pdffitwindow=false,     % window fit to page when opened
    pdfstartview={FitH},    % fits the width of the page to the window
    pdftitle={Syllabus: Public Opinion, Political Psychology, and Citizenship},    % title
    pdfauthor={Thomas J. Leeper},     % author
    pdfsubject={Political Science},   % subject of the document
    pdfkeywords={politics} {public opinion} {political psychology}, % list of keywords
    pdfnewwindow=true,      % links in new window
    pdfborder={0 0 0}
}

\begin{document}
\nobibliography*
\faketableofcontents

\begin{center}
{\Large
\noindent \textbf{GV4J3\\ Public Opinion, Political Psychology, and Citizenship}
}
\end{center}
\vspace{1em}

\noindent
Thomas J. Leeper\\
Office: CON 3.21\\
Office hours: By appointment via LfY\\
Email: \href{mailto:t.leeper@lse.ac.uk}{t.leeper@lse.ac.uk}

\vspace{1em}

\noindent Course website: \url{https://moodle.lse.ac.uk/course/view.php?id=5109}\\
Reading list: \url{https://library-2.lse.ac.uk/e-lib/e_course_packs/GV4J3/GV4J3_64769.pdf}\\

\noindent The purpose of this course is to explore issues related to public opinion, including what opinions are and how they are formed, what factors do and do not influence opinion development and change, how opinions drive citizens' political thinking and behaviour, and what implications these psychological processes have for the role of public opinions in democratic government. Students will leave the course with a thorough theoretical understanding of political opinions, their origins, and their possible effects through exposure to philosophical perspectives, contemporary case studies, and a broad set of empirical research.

\section{Objectives and Evaluation}
After this course, students should be able to:
\begin{enumerate*}
\item Explain what opinions are and how they are formed.
\item Describe properties of public opinion at the individual and aggregate levels.
\item Evaluate political psychological theories and normative arguments about public opinion.
\item Evaluate the quality of empirical public opinion research.
\item Explain and apply quantitative and qualitative methods to the study of public opinion.
\end{enumerate*}

\noindent It is important to note that this is a research seminar that is informed by original political science research and evaluated through participants' own original research paper. It may be useful to think of the course as a ``mini-dissertation'' project.

\section{Summative Assessment: Exam Paper}

The exam for the course is an independent research paper of approximately 5,000 words that:

\begin{enumerate}
\item addresses an important political science question related to public opinion, political psychology, or political behaviour,
\item offers a theoretical contribution toward understanding that question, and 
\item reports an original empirical analysis that tests that theory.
\end{enumerate}

Original data analysis (and possibly data collection) are required, though the particular form of the empirical component can be qualitative, quantitative, or both. Some examples of empirical projects include: an original survey and/or experimental data collection, a pilot test of a proposed research design (and the description of a more complete empirical design), qualitative data analysis (such as focus groups, semi-structured interviewing, content analysis, etc.), the analysis of existing public opinion data (e.g., surveys, cross-national comparisons, election results, etc.), or some mix of these. While it is not expected that students conduct a large-scale study, they must conduct some novel data collection and/or analysis.

The exam will be marked according to guidelines available in \href{http://www.lse.ac.uk/government/degreeProgrammes/programmes/masters/MSc-Handbook-2015-6.pdf}{the Government Department MSc Handbook}. Marks are assigned according to the conventional LSE scale and written feedback will be provided on the assessed essay.

\vspace{1em}

\noindent The exam is due \textbf{XX XX, 2017 at 5pm}.


\section{Formative Assessment}

Formative assessment consists of (1) a 2-page written proposal for the final essay and an associated literature review, (2) four out-of-class problem sets submitted near the beginning of the term, and (3) in-class discussion activities.

\subsection{Research Proposal}

In preparation for the final exam, students will prepare a short, 2-page proposal to be submitted in Week 5 of Lent Term that outlines a possible project for the assessed essay. This document should state a research topic and clear research question, make reference to relevant theoretical and empirical literature, and propose a basic design for addressing the question. Students should focus on one topic, but can present up to two distinct ideas if they are undecided about what to do. The proposal should be uploaded to Moodle.

Students should then meet one-on-one with the instructor during Week 6 to discuss the ideas, receive feedback, and make plans for the final paper. Once a topic is agreed, student should use Week 6 to complete an annotated literature review of 5--10 relevant studies (from reading list and elsewhere) that motivate the final project and upload it to Moodle.

Once finalized, 3--4 students per week will be asked to briefly present their projects for peer feedback during class meetings in Weeks 7--10 of term. These presentations should be oral and last about 5 minutes.

\subsection{Problem Sets}

Given the combination of an assessed essay as the sole summative assessment, a relatively short term (10 weeks), and the varied backgrounds of students enrolled in the course, short problem sets applying different research methods in public opinion are due in the first four weeks of the course (Weeks 2--5). These provide an opportunity to both gain methodological competence to critique readings in the course and prepare the final exam project.

\begin{center}
\begin{tabular}{lll} \hline
\textbf{Problem Set} & \textbf{Assigned} & \textbf{Due Date} \\ \hline
Week 2: Interviewing & Jan. xx & Feb. xx \\
Week 3: Trends and toplines & Jan. xx & Feb. xx \\
Week 4: Correlation and regression & Jan. x & Jan. xx \\
Week 5: Experimentation & Feb. xx & Feb. xx \\ \hline
\end{tabular}
\end{center}

The problem sets take the form of ``replication'' activities, in which the data from a published research article is made available and students are asked to reproduce the results of the paper from the original data and explain the logical of the underlying methods. The problem sets are mandatory but are not marked. Please treat them as an opportunity to self-evaluate and learn and to approach the instructor with any hesitations you may have. Collaboration is allowed, but each student should submit an individual assignment. Marking rubrics will be provided.

\subsection{Discussion Activities}

The course will primarily involve student-led discussions with the exception of a few lecture elements surrounding methodological issues in public opinion research. The course is structured as a ``reading group,'' where every student is expected to have read all assigned readings and should be able to summarize and critique each reading if asked to do so.

Given the discussion format of the course, some preparation is required on the part of each student for each seminar meeting:

\begin{itemize}
\item Every week, every student must post 1 or 2 discussion questions to Moodle based upon the topic of the week and the assigned readings. These are due by Thursday at 17:00 (5:00pm) prior to class.
\item Each student will for one week's meeting serve as discussion co-leader (there will be 2-3 discussion leaders per week). This requires them to review the discussion questions posted to Moodle and facilitate a conversation around the assigned readings. It also requires a short writing task, described next. 
\end{itemize}

Each discussion leader should write a short (2-3 page; max 1500 words) reflection paper that synthesizes course readings for that week, raises questions unanswered by the texts, or proposes avenues for further theoretical development and/or empirical research. These should be opinionated essays that make an argument about public opinion; they should not be simply summaries of the readings.

Class will start each week with these students discussing their papers. All students should read the papers before class.





\clearpage
\section{Course Outline}

Class will meet at the following times and locations:

\begin{itemize}
\item Friday 15:00-17:00 (XXXX) in LT Weeks 1-10
\end{itemize}


\noindent The general schedule for the course is as follows. Details on the readings for each week are provided on the following pages.

\secttoc



\clearpage
\subsection{Week 1: Public Opinion and Democracy (Jan. 15)}
\reading[Chapters 7 (Available at: \url{https://library-2.lse.ac.uk/e-lib/e_course_packs/GV4J3/GV4J3_64769.pdf}) from ]{Dahl1989}
\reading[Chapters 3 (Available at: \url{https://library-2.lse.ac.uk/e-lib/e_course_packs/GV4J3/GV4J3_64770.pdf}) from ]{Herbst1995}
\reading{Mansbridge2003}
\reading{Druckman2014}

\seealso
\reading{Riker1988}
\reading{Lippmann1922}
\reading{Lippmann1928}
\reading{Dahl2006}
\reading{Pitkin1967}
\reading{Disch2011}
\reading{Downs1957}
\reading{HibbingTheiss-Morse2002}
% 2015 article critical of empirical study of representation
% Druckman PolComm comment
\reading{JacobsPage2005}
\reading{BachrachBaratz1962}
\reading{Fishkin2006}
\reading[Chapter 8 from]{Schattschneider1975}
\reading{McCloskyHoffmanOHara1960}
\reading{Erikson1978}

\clearpage
\subsection{Week 2: What are opinions? (Jan. 22)}
\reading[Chapter available at \url{https://library-2.lse.ac.uk/e-lib/e_course_packs/GV4J3/GV4J3_64771.pdf} from ]{EaglyChaiken1998}
\reading[Chapter 3 (Available at  \url{http://ebooks.cambridge.org.gate2.library.lse.ac.uk/ebook.jsf?bid=CBO9780511818691}) from ]{Zaller1992}
\reading{Evans2008}

\seealso
\reading{DruckmanLupia2000}
\reading{EaglyChaiken2007}
\reading{EaglyChaiken1993}
\reading{Fazio2007}
\reading{Ajzen2001}
\reading{PettyKrosnick1995}
\reading{Hovland1953}
\reading{McGuire1969}
\reading{Katz1960}

\clearpage
\subsection{Week 3: Measuring Opinions (Jan. 29)}

\reading{TourangeauRasinski1988}
\reading{BishopTuchfarberOldendick1986}
\reading{SchuldtKonrathSchwarz2011}
\reading{ZallerFeldman1992}
\reading{Kam2007a}

\seealso
\reading{Grovesetal2009}
\reading{Lohr2009}
\reading{Berinskyetal2011}
\reading{SchumanPresser1981}
\reading{SchaefferPresser2003}
\reading{KrosnickJuddWittenbrink2005}
\reading{RevillaSarisKrosnick2013}
\reading{YeagerLarsonKrosnickTompson2010}
\reading{Krosnicketal2002}
\reading{Brady2000}
\reading{Feldman1989}
\reading{AlvarezFranklin1994}

\clearpage
\subsection{Week 4: Opinion Dynamics and Change (Feb. 5)}

\reading{LaxPhillips2009a}
\reading{DruckmanLeeper2012b}
\reading{PageShapiroDempsey1987}
\reading{MacKuenEriksonStimson1989}
\reading{Wlezien2012}


\seealso
\reading{Callegaroetal2014}
\reading{AndresGolschSchmidt2013}
\reading{DruckmanLeeper2012a}
\reading{Gilens2001}
\reading{AnsolabehereRoddenSnyder2008}
\reading{Sanders2012}
\reading{JohnstonBrady2002}
\reading{PageShapiro1992}
\reading{Wood2000}
\reading{MulliganGrantBennett2013}

\clearpage
\subsection{Week 5: Opinion Responsiveness (Feb. 12)}

\reading{ChongDruckman2007a}
\reading{Kuklinskietal2000}
\reading{Barabasetal2015}
\reading{DruckmanNelson2003}
\reading{PettyCacioppo1986}

\seealso
\reading{DelliCarpiniKeeter1997}
\reading{GainesKuklinskiQuirk2007}
\reading{Holland1986}
\reading{LodgeMcGraw1995}
\reading{LodgeSteenbergenBrau1995}
\reading{Iyengar1990}
\reading{LauRedlawsk2001}
\reading{GronlundMilner2006}
\reading{TilleyWlezien2008}
\reading{Mutz1992}
\reading{HuckfeldtSprague2006}
\reading{Nickerson2008}
\reading{BennettIyengar2008}
\reading{ChongDruckman2007b}
\reading{DruckmanFeinLeeper2012}
\reading{Gerberetal2011}

\clearpage
\subsection{Week 6: Reading Week (Feb. 19) - no class meeting}

There will be no class meeting on February 19 (LT Week 6) due to LT Reading Week. By this point in the course, students should have an idea in mind for their final exam essay topic or even a relatively elaborated proposal for the exam paper. To obtain feedback on these ideas, students will be assigned to groups of 3-4 students. Groups will meet during Reading Week (at a time convenient for all involved) and provide peer feedback on these formulations. Written proposals should be distributed to group members in advance of the meetings via Moodle.


\clearpage
\subsection{Week 7: Predispositions and Constraint (Feb. 26)}
\reading{JostFedericoNapier2009}
\reading{Feldman1988}
\reading{AlfordFunkHibbing2005}
\reading{Gerberetal2011b}
\reading{Mutz2002}

\seealso
\reading{BrewerGross2005}
\reading{Converse1964}
\reading{JenningsNiemi1975}
\reading{Lacy2001a}
\reading{CharneyEnglish2012}
\reading{LopezMcDermott2012}
\reading{McPhersonSmithLovinCook2001}
\reading{PeffleyHurwitz1985}

\clearpage
\subsection{Week 8: Heuristics and Cognitive Biases (Mar. 4)}

\reading{LauRedlawsk2001}
\reading{MarksMiller1987}
\reading{Petersenetal2011}
\reading{Healy2010}
\reading{BraderValentinoSuhay2008}

\seealso
\reading{Chaiken1980}
\reading{DanceySheagley2012}
\reading{TverskyKahneman1974}
\reading{KuklinskiHurley1994}
\reading{Hobolt2007}
\reading{Arceneaux2007}
\reading{ValentinoHutchingsWhite2002}
\reading{Lupu2013}
\reading{Huddyetal2005}

\clearpage
\subsection{Week 9: Motivated Reasoning (Mar. 11)}

\reading{TaberLodge2006}
\reading{DruckmanBolsen2011}
\reading{SlothuusdeVreese2010}
\reading{LeeperSlothuus2014}

\seealso
\reading{Kunda1990}
\reading{MoldenHiggins2004}
\reading{RedlawskCivettiniEmmerson2010}
\reading{Druckman2012}
\reading{DruckmanPetersonSlothuus2013}
\reading{BolsenDruckmanCook2013}
\reading{HartNisbet2011}
\reading{Groenendyk2013}
\reading{Cohenetal2007}
\reading{KruglanskiWebster1996}
\reading{Dittoetal1998}

\clearpage
\subsection{Week 10: Political Identity (TBD)}

\reading{Greene1999}
\reading{Huddy2001}
\reading{Klar2013}
\reading{IyengarSoodLelkes2012}

\seealso
\reading{KnowlesGardner2008}
\reading{KangBodenhausen2015}
\reading{WhiteLairdAllen2014}
\reading{GreenPalmquist1994}


\bibliographystyle{plain}
\nobibliography{References}

\end{document}
