\documentclass[12pt,a4paper]{article}
\usepackage[margin=1in]{geometry}
\usepackage{graphicx,setspace,hyperref,amsmath,amsfonts,multirow,ccaption,mdwlist,comment}
% mini table of contents
\usepackage{minitoc}
\dosecttoc % make section toc
\setcounter{secttocdepth}{2} % subsection depth
\renewcommand{\stctitle}{} % no title
\nostcpagenumbers

% optionally include commented environments
\excludecomment{lessonplan}
\excludecomment{finalexam}

\setlength{\marginparwidth}{.5in}
\usepackage{natbib}
% Two lines to create in-text full citations for a syllabus
% And comment out my other standard bibtext commands
\usepackage{bibentry}
\newcommand{\reading}[2][]{\noindent -- {#1 }\bibentry{#2}.\vspace{.25em}\\}
\newcommand{\seealso}{\noindent \emph{See Also:}\\}
\newcommand{\topic}[1]{\noindent \textbf{#1}\\}
\usepackage[T1]{fontenc}
\usepackage{lmodern}
\hypersetup{
    bookmarks=true,         % show bookmarks bar?
    unicode=false,          % non-Latin characters in Acrobat’s bookmarks
    pdftoolbar=true,        % show Acrobat’s toolbar?
    pdfmenubar=true,        % show Acrobat’s menu?
    pdffitwindow=false,     % window fit to page when opened
    pdfstartview={FitH},    % fits the width of the page to the window
    pdftitle={Syllabus: Public Opinion, Political Psychology, and Citizenship},    % title
    pdfauthor={Thomas J. Leeper},     % author
    pdfsubject={Political Science},   % subject of the document
    pdfkeywords={politics} {public opinion} {political psychology}, % list of keywords
    pdfnewwindow=true,      % links in new window
    pdfborder={0 0 0}
}

\title{Public Opinion, Political Psychology, and Citizenship}
\author{Thomas J. Leeper\\
Department of Government\\
London School of Economics and Political Science}

\begin{document}
\nobibliography*

\maketitle

\faketableofcontents

\section{Overview}


\section{Practical Matters}

The expectations for this course are that students (1) participate actively, regularly, and positively in classroom discussions (which will constitute the bulk of the course's content), (2) lead discussion on at least one day of class, and (3) complete a written exam answering questions raised by the course using relevant theoretical and empirical literature. Toward the first expectation, students should read the assigned reading ahead of the day on which they are assigned and have at least two questions in mind that were provoked by those readings that might be answered in class or serve as a topic for discussion. Toward the second end, students will sign up for one or more weeks to lead discussion (on the first day of class), which will also involve writing a short (one-page) response essay to structure that discussion. On their assigned week, students can structure class discussion however they so choose, but should use submitted discussion questions where useful.

\section{Objectives and Evaluation}
After this course, students should be able to:
\begin{enumerate*}
\item Explain what opinions are and how they are formed.
\item Describe properties of public opinion at the individual and aggregate levels.
\item Explain different conceptualizations of political representation and their empirical implications.
\item Evaluate arguments about the proper role of public opinion in democracy and government.
\item Apply knowledge of opinions and opinion measurement to the evaluation of survey public opinion research.
\end{enumerate*}

\clearpage
\section{Course Outline}
The general schedule for the course is as follows. Details on the readings for each week are provided on the following pages.
\secttoc
\clearpage

\subsection{}

% - Week 1 (Jan. 15): What are opinions?
% - Week 2 (Jan. 22): Measuring opinions (and survey methods)
% - Week 3 (Jan. 29): Opinion dynamics and change (and macro opinion analysis)
% - Week 4 (Feb. 5): Information and opinion responsiveness (and experimental methods)
% - Week 5 (Feb. 12): Predispositions and Constraint
% - Week 6: Reading Week peer feedback sessions (see below)
% - Week 7 (Feb. 26): Motivated reasoning
% - Week 8 (Mar. 4): Heuristics and cognitive biases
% - Week 9 (Mar. 11): Representation
% - Weeks 10-11: Revision and feedback sessions



% load bibtext, but don't generate a bibliography
\bibliographystyle{plain}
\nobibliography{References}

\end{document}